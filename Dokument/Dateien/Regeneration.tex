\chapter{Regeneration}
Bei allen Mechaniken sollte der Fakt bedacht werden, dass wir ein Rollenspiel mit Fokus auf die Geschichte und Entwicklung wollen. Der Kampf soll keine primäre Rolle spielen wie bspw. in Skyrim.

\section{Leben}
\begin{itemize}
	\item der Spieler erhält eine sehr kleine Auto-Regeneration. Die natürliche Selbstheilung. In ruhigen Momenten (z.B. Schlaf) regeneriert er pro Stunde zB 2 Prozent seines maximalen Lebens. So können kleine Kampfwunden auch von alleine heilen.
	\item weitere Möglichkeiten zur Heilung: Das Aufsuchen von Heilern und Essen.
	\item Heiler können einen nach entsprechender Gebühr und in einer Zeitspanne, die abhängig von ihrer Expertise ist, komplett voll heilen
	\item Durch Essen lässt sich maximal 60 Prozent des Effektivschadens am Gesamtleben heilen. Verliert der Spieler mehr als 60 Prozent Leben, dann ist die Verletzung entsprechend intensiv und signifikant und soundso viele Prozentpunkte müssen von einem Heiler behandelt werden (permanenter Schaden bis geheilt).
\end{itemize}

\section{Mana}

\section{Hunger}
\begin{itemize}
	\item Man muss täglich wenigstens einen Happen zu sich nehmen, sonst bekommt man Debuffs (auf höheren Schwierigkeitsstufen mehr zB dreimal täglich ausreichend viel, auf niedrigeren zB gar nichts). Je länger, desto heftiger die Debuffs
	\item Man hat einen vergrößerten, aber nicht überdimensionierten Magen. Das bedeutet: Stopft man viel zu viel Essen in sich hinein, bekommt man ab einer gewissen Schwelle nach den 100 Prozent "Magen voll" (zB bei 150 Prozent) Debuffs, weil man Bauchschmerzen hat. Stopft man dann noch mehr in sich rein (zB bis 200 Prozent), dann muss man sich übergeben und verliert alles, was noch im Magen ist, und nimmt zudem etwas Schaden (durch das heftige Kotzen). Zudem verweigert der Spielercharakter für zB eine Woche das Essen des i-Tüpfelchens (was das Brechen dann ausgelöst hat), weil ihm schon bei dem Gedanken schlecht wird. 
	\item Die Magen-Anzeige im Spiel umfasst natürlich nur 0-100 Prozent
	\item Regeneration durch Essen erfolgt sekündlich immer um den gleichen Betrag (nicht Prozent!) - wie lange wird durch das jeweilige Essen festgelegt. So wird eine Gurke zB nur für 1s heilen, eine Buttercremetorte oder ein Fasanenbraten hingegen für 30s. Den Magen füllen sie ebenso unterschiedlich. 
	\item Idee für Erweiterung des Soldaten-Regenerationsskills: ein Punkt kann sein, dass er gelernt hat, sein Essen bedachter zu verspeisen und besser zu verwerten. Das führt zu einer Erhöhung der Heilung pro Sekunde
	\item Idee: jeder/einer/zwei der Start-Charas könnte eine Unverträglichkeit bzw "schmeckt nicht" haben. Das führt dazu, dass man mit diesem Chara dieses Essen nicht nutzen kann bzw. im zweiten Fall dieses Essen nur einen minimalen Bruchteil der Regeneration bietet. Und zB nach einer gewissen Menge, die man das in sich stopft (unabhängig vom 150 Prozent Magen), dann auch Debuffs dazu kommen\\
	man kann das natürlich festschreiben oder wir lassen den Spieler am Anfang des Spiels selbst entscheiden, was es wird, indem das ein Teil einer Kinder-Quest ist
\end{itemize}