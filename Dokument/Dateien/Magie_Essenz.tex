\chapter{Magie}
\href{http://www.weltenbau-wissen.de/2015/12/magie-weltenbau-magiesystem-mystik-wissenschaft-teil-1/}{Magie soll Mystik behalten}\\
\href{http://www.weltenbau-wissen.de/2016/01/6-konsequenzen-magie/}{Konsequenzen von Magie}\\
\href{https://meisterperson.wordpress.com/2016/05/05/magie-fortschritt/?pk_campaign=pifeed&pk_kwd=magie-fortschritt}{Magie und Fortschritt}

\subsection{Was ist Magie?}
\begin{itemize}
	\item Magie ist eine besondere Form der Energie und entsteht als "Nebenprodukt" bei Energieumwandlungen. Ebenso kann Magie in andere Energieformen umgewandelt werden
	\item in Lebewesen entsteht daher besonders viel Magie, weil in den Zellen permanent ganz viel Energieumwandlung erfolgt
	\item in der unbelebten Natur entsteht Magie nur bei Dingen wie Steinschlägen, Lawinen, Vulkanen, Katastrophen etc
	\item Lebewesen können die Energie in ihrer Form der Magie halten (WIE?). Die Menge ist artabhängig und angeboren. Ist die Menge der haltbaren Magie überschritten, so wandelt sich alles überschüssige wieder in andere Energieformen um bzw diffundiert aus dem Körper heraus in die Welt und wandelt sich da um. Die so haltbare Magie wird auch als Mana oder Manapool bezeichnet \\
	Verbildlichung:  Zisterne, die sich mit Wasser füllen lässt. Wenn sie voll ist, läuft sie einfach über und das Wasser verteilt sich in der Gegend
	\item manche Lebewesen haben im Laufe der Evolution Mechanismen entwickelt, um die ihnen innewohnende Magie in gewollter Art und Weise umzuwandeln bzw einzusetzen. So gibt es Pflanzen, die ohne spezielle Organe Elektroschocks, Strahlung, Vibration oder Licht nutzen, um Fressfeinde los zu werden und Bestäuber anzulocken. Oder Predatoren, die mit Temperaturbeeinflussung oder Licht jagen. Auch der Mensch hat diese Fähigkeit entwickelt.
	\item der Mensch hat später festgestellt, dass er mithilfe seines Willens und Konzentration dazu in der Lage ist, diese Möglichkeiten auszubauen und zu verstärken. Dies hat ihm einen nötigen Vorteil gegeben, sich durchzusetzen. Dabei entwickelten sich jedoch bei den verschiedenen Menschenarten unterschiedliche Ausprägungen bzw. gingen teilweise einst vorhandene verloren
	\item bei dem Einsetzen von Magie kann das Lebewesen dabei auf ein Ziel ausgerichtet die Magie nutzen, um aktuelle Zustände zu ändern. Es ist ein halbwegs äquivalenter Austausch von Magie zu benötigter Energie für diese Änderung nötig
	\item alle "Magiebegabten" aka Magie-nutzen-könnende Lebewesen setzten sie intuitiv wie Reaktionen oder als Unterstützung ein. Auf diesem Level wird sie auch von den meisten Menschen beherrscht. Nur Personen, die sich intensiv mit ihren Fähigkeiten auseinandersetzen, viel Meditieren und Verstehen lernen, nur diese Personen lernen, das Tauschverhältnis zu reduzieren und so mit ihrem Manapool mehr und Stärkeres bewirken zu können. Dabei sind ihnen nur durch ihre Gene und durch ihre Vorstellungskraft und Konzentrationsfähigkeit Grenzen gesetzt
	\item einfache "Spells" (nicht darunter bekannt), also einfaches Wirken  von kleinen Zaubern, das lernen Kinder in ihrer Kindheit von Eltern oder Verwandten oder durch neugieriges Experimentieren und konzentrieren
	\item im Allgemeinen ist es recht einfach, den eigenen Körper und die Umgebung zu beeinflussen. Die Beeinflussung anderer Lebewesen, erst recht mit steigendem Selbstbewusstsein und Willen, ist hingegen recht schwer und nur von spezialisierten Magiern möglich
\end{itemize}


\subsection{Die Genetik der Magie}
\begin{itemize}
	\item Magie wird vererbt und zwar (der Einfachheit halber) mendelsch. D.h. es existeren in jedem Tier (bei Pflanzen ist das alles komplizierter) je ein Allel von der Mama und ein Allel vom Papa durch welches festgelegt ist, welche Magie man ausprägen kann. Wenn also beide Eltern homozygot (aka beide Allele gleich) Elementaristen sind, dann kann das Kind auch nur ein Elementarist sein.
	\item kurz gesagt: seltene Magie Arten sind rezessiv, häufige dominant. Ansonsten gibt es Magie-Arten, die sich so aufgrund der Kombination zweier dominanter Allele ausprägen
\end{itemize}

\subsection{Magiearten}
\subsubsection{Temperatur}\label{sec:temperaturmagie}
\begin{itemize}
	\item diese Fähigkeit erlaubt es einem, die thermische Energie der Umgebung zu kontrollieren. Dies bedeutet sowohl das Erhitzen bis zum Brennen als auch das Abkühlen bis zum Gefrieren
	\item auf intuitivem Level bedeutet es idR, dass diese Menschen im Winter und Sommer weniger Temperaturprobleme haben, als auch die Kochstelle mit ein bisschen Konzentration entflammen können.
	\item im weiteren Verlauf der Beherrschung sind zB: Feuerbälle, Flammenwände, kochen des Gegenübers mit dessen eigenen Blut, einfrieren, Glatteis, löschen von Feuer
\end{itemize}

\subsubsection{Wind oder Druck}\label{sec:druckmagie}
\begin{itemize}
	\item diese Leute können den lokalen Druck beeinflussen und damit auch Wind erzeugen
	\item intuitiv ist darin erstmal nur das wegstoßen von Leuten/Gegenständen sowie des eigenen Körpers inbegriffen. Oder auch ein laues Lüftchen zu erzeugen
	\item später: extrem längere und weitere Sprünge durch Windunterstützung; schweben; etwas oder jemanden zu Boden oder an die Wand drücken; den Druck im Kopf erhöhen, bis er platzt oder auch den Blutdruck; Dinge mit Druck zerquetschen, Windhosen erzeugen, Regenwolken lenken
\end{itemize}

\subsubsection{Vibration}\label{sec:vibrationsmagie}
\begin{itemize}
	\item Erlaubt einem sowohl die Vibrationen in seiner Umgebung verstärkt zu spüren als auch selbst Vibrationen zu erzeugen
	\item Intuitiv bedeutet das eine verbesserte Wahrnehmung von Geräuschen, vielleicht bis hin zu einer Fledermausähnlichen Echolot-Technik. Auch das Erzeugen kleinerer Geräusche mittels Vibration wäre möglich.
	\item Im weiteren Verlauf kann hiermit die Umgebung extrem verstärkt wahrgenommen werden, bis hin zum erspüren von weit entfernten Lebewesen/Gegenständen, Erzeugung von lauten Geräuschen oder Zerstörung von Gegenständen/Gewebe durch Vibration
\end{itemize}

\subsubsection{Licht und Dunkelheit} \label{sec:lichtmagie}
\begin{itemize}
	\item Ermöglicht die Kontrolle von Licht im Allgemeinen. Bei Steigerung auch die Kontrolle der Abwesenheit von Licht (Dunkelheit).
	\item Intuitiv wäre es denkbar kleinere Lichtquellen zu erzeugen. Möglicherweise bis hin zu kleineren Illusionen
	\item Ausgebaut würde das die komplette Kontrolle über die An- bzw. Abwesenheit von Licht bedeuten, sowie das erschaffen von lebensechten Illusionen, das Blenden von Gegnern und die eigene Unsichtbarkeit (Licht um einen herum leiten).
\end{itemize}

\subsubsection{Verhärtung}\label{sec:haertungsmagie}
\begin{itemize}
	\item Das Verhärten von verschiedensten Dingen. Angefangen beim eigenen Körper bis hin zu externen Objekten. Genauso andersherum das Erweichen von Dingen
	\item das Verhärten beruht physikalisch gesehen darauf, die Atomverbindungen zu anderen Atomen zu stärken, so dass diese nicht zerstört werden können
	\item Intuitiv das Verhärten des eigenen Körpers bis hin zu Objekten durch Handauflegen
	\item Ausbaubar, bis hin zu stahlharter Verstärkung von Dingen, unter anderem Luft (Wände in der Luft). Außerdem ist es möglich Dinge \& Gewebe auf ihrer Molekularen Ebene auseinander zu reißen
\end{itemize}

\subsubsection{Proliferation}\label{sec:proliferationsmagie}
\begin{itemize}
	\item Das Beeinflussen der Reaktionsgeschwindigkeit
	\item Intuitiv bedeutet das, dass diese Menschen etwas kürzer leben, aber dafür schneller heilen als der Durchschnitt, da die Regenerationsprozesse in ihrem Körper beschleunigt sind. Weiterhin kann kurzfristig die Entwicklung/das Wachstum von Pflanzen beeinflusst werden
	\item Ausbaubar wäre das zB in Richtung von explosiv wachsendem Gewebe (Tumore), altern lassen von Lebewesen, Beschleunigung der Regeneration von Verletzungen etc
\end{itemize}

\subsubsection{reine Energie}\label{sec:energiemagie}
\begin{itemize}
	\item ...
\end{itemize}

\subsubsection{Strahlung}\label{sec:strahlungsmagie}
\begin{itemize}
	\item Ermöglicht das Aussenden und Absorbieren von hochenergetischer Strahlung (energiereicher als Licht, aka UV, alpha-, beta-, gamma-Strahlung)
	\item Intuitiv bedeutet das, dass diese Menschen keinen Sonnenbrand bekommen
	\item Ausbaubar ist das dazu, andere krank werden zu lassen (Mutationen etc) im kleinen Stil wie durch Tumore oder im großen Stil durch die Strahlungskrankheit
\end{itemize}

\subsubsection{Elektrizität und Magnetismus}\label{sec:elektromagie}
\begin{itemize}
	\item Das Beeinflussen elektrischer und magnetischer Ströme/Ladungen
	\item Intuitiv das Abgeben eines elektrischen Schocks oder kleine Bewegungen von magnetischem Material
	\item Ausbaubar bis in zur Beeinflussung von elektrischen Strömen an Nervenzellen (Signalweiterleitung mittels Neuronen), (ent-)magentisierung von Metallen, Kontrolle über die Bewegung von Metallen, Blitze ableiten, WIEDERBELEBUNG!? (unter entsprechenden Umständen)
	\item Wiederbelebung: so weit kam nur ein Zauberer, welcher schon verstorben ist: konnte manchmal Leute zurück ins Leben bringen, wenn es noch dicht genug am Todeszeitpunkt dran war -> war sozusagen ein Defibrillator
\end{itemize}

\subsubsection{Absorption oder Nullmagie}\label{sec:nullmagie}
\begin{itemize}
	\item eine besondere Art von Magie ist die "Nullmagie" (andere Namen möglich). Diese Magieart ist sozusagen Anti-Magie, denn der Träger ist nicht fähig, Magie zu wirken, sondern nur zu absorbieren und damit ihre Effekte zu verhindern. Ebenso kann keine Magie auf ihn gewirkt werden. Er neutralisiert und absorbiert Magie - kann diese absorbierte Magie aber auch nicht nutzen, sondern die verschwindet gleich wieder aus seinem Körper
	\item zum aktuellen Zeitpunkt nicht existent. Entsteht erst bei der Gottwerdung in den Überlebenden. Von da an rezessiv vererbbar
\end{itemize}


\chapter{Essenz}
\subsection{Was ist Essenz?}
\begin{itemize}
	\item Lebewesen besitzen eine Essenz, eine Art Seele (wird im Folgenden als Seele bezeichnet, um die Erklärung einfacher zu halten). Diese Seele umfasst das, was dieses Lebewesen "sein eigen" nennt im Sinne des Körpers, der Existenz. Das ist im Normalfall der Körper. Könnte aber auch eine Prothese am Körper umfassen, wenn das Lebewesen diese als Teil seines Körpers begreift.
\end{itemize}