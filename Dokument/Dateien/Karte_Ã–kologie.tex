\section{Karte \& Ökologie}
\subsection{Sonnensystem}
\subsubsection{Welt + Zwilling}

\subsection{Welt}
\subsection{Kontinent}
\subsubsection{Gebirgszug}
\subsubsection{Gigantus Wald}
\begin{itemize}
	\item wie bekannt, können auch Pflanzen die ihnen innewohnende Magie nutzen. Eine Art (!!) von Bäumen haben folgendes entwickelt: sie ermöglichen mittels der Magie die Wasserversorgung der oberen Baumabschnitte gegen die Gravitation. Das Größeneinschränkende Element normaler Bäume ist damit weg. Daher konnten die Bäume extrem groß werden, bevor weitere Prozesse ihr Wachstum stoppten und hatten damit den Platz an der Sonne sicher
	\item  diese Garganten (Working title) erreichen Höhen von 500m und ähnlich wie aus dem Dschungel bekannt, bilden sich somit verschiedene vertikale Lebensbereiche
	\item der Boden ist bedeckt mit allerlei Gehölz und Sträuchern, die wenig Sonnenlicht brauchen, und ganz vorne voran riesigen Pilzen, die in Symbiose mit den riesigen Bäumen ebenfalls ein gigantisches unterirdisches Netzwerk bildeten und nun gigantische Fruchtkörper (:mushroom:  das da) ausbilden können
	\item die ersten kräftigen Äste der Garganten haben stark grüne, mit viel Chlorophyll gefüllte Blätter, die jegliches Licht, das durch die oberen Schichten kommt, aufnehmen kann. Für unsere Verhältnisse stehen die Bäume zwar ewig weit auseinander, doch auf ein normales Größenverhältnis reduziert nicht. Ihre Äste können sich an den Spitzen erreichen und bilden so mittels der Äste und den riesigen Laubblättern eine Art zweiten Boden, beginnend in 200m Höhe
	\item durch die Ausmaße der Äste sammelt sich Erde hier und dort und es wachsen normale Bäume auf dieser Ebene, allerdings nicht zu weit vom Stamm der Garganten entfernt. Ebenso finden sich hier normale Büsche etc. Schlingpflanzen ziehen sich um die dicken Äste und es bildete sich so im Laufe der Zeit in dicker "Boden" - mit unerwartete Löchern hier und da
	\item Da wir uns hier schon in der Krone der Bäume befinden, ist ab dieser zweiten Schicht ein langsamer Verlauf in den folgenden 300m: Die Baumkrone begünstigt in ihren Ausmaßen das Wachsen anderer Bäume, Sträucher, Kräuter etc, insbesondere parasitischer Lebensformen wie Schlingpflanzen und Misteln. Auch einige "normale Bäume" wachsen nicht nur auf den Erdansammlungen, sondern auch in das Holz der Äste hinein. Insgesamt bildet sich so wie eine Art Gerüst, welches aus festen "Platformen", aus festen und wackligen Wegen horizontal, vertikal und schräg, aus Bereichen lockerer Vegetation (wie Lianen) und aus "Lichtungen" (3D, nicht 2D) besteht.
	\item nach oben hin werden die Blätter der Garganten immer lichtdurchlässiger. Die untersten Blätter sind mehrere Meter dick und dunkelgrün vor Chlorophyll. Die obersten sind nur ein paar Zentimeter dick, kleiner, leicht hellgrün gefärbt und ansonsten Lichtdurchlässig. Das sorgt dafür, dass das Licht bis unten hin durchkommt und effizient genutzt werden kann, um den riesigen Baum (und seine Parasiten) zu versorgen
	\item ebenso wie die Fauna, hat sich hier auch viel Getier angesammelt und angepasst in den verschiedenen Stufen. So nisten bestimmte Vögel nur in den oberen 50m der Baumkronen. Auch ein paar Menschen fanden sich hier vor langer Zeit ein und haben sich an das Leben in den Baumkronen (von 200-350m etwa) angepasst. Es ist die Heimat der \npref{rasse:sylvan}.
\end{itemize}

\subsection{Land}
\subsection{Tal}
\subsubsection{Teich 1}
\subsubsection{Teich 2}
\subsubsection{Großer Wald}
\subsubsection{Klippe am Eingang}

