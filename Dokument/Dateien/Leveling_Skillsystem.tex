\chapter{Leveling und Skills}
\section{Leveling}
\begin{itemize}
	\item Es gibt verschiedene Arten von Skills. Manche können erlernt und etwas gesteigert werden, ohne dass dafür besondere Dinge oder Zeit nötig sind (zB Regeneration). Die anderen wiederum haben ein bestimmtes System aus Selbst-Lernen, Lehrmeistern und Üben
	\item Ich erkläre das System am Beispiel des Schlösserknackens unter der Annahme, dass wir 5 Stufen der Beherrschung eines Skills einführen (zB Neuling, Lehrling, Adept, Meister, Großmeister)
	\item Zu Beginn hat man die Fähigkeit nicht. Es gibt nun zwei Wege, sie zu erwerben:\\
	1. Selbstlehrend. Dabei setzt man sich einfach mal an ein zu knackendes Schloss und egal ob man es schafft oder nicht, man ist nun in der ersten Stufe, Neuling. \\
	2. Lehrmeister. Man kann auch zu einem entsprechenden Lehrmeister gehen (z.B. ein Schlosser) und ihn bitten, einem die Grundlagen beizubringen.
	\item um nun die Fähigkeit zu verbessern sind zwei Abschnitte relevant: Das Üben und der Aufstieg in die nächste Stufe (und damit verbunden neue Fähigkeiten-Möglichkeiten)
	\item erstmal zum Aufstieg. Um die nächste Stufe zu erreichen, ist IMMER ein Lehrmeister (oder auf den niedrigen Stufen meinetwegen auch Lehrbücher) nötig. Dies sorgt auch dafür, dass wir die Relevanz von Geld in unserem Spiel erhöhen.
	In unserem Beispiel würde einem der Schlosser Erläuterungen zu kompizierteren Schlössern geben und sie zeigen.
	\item das Üben ist nötig, bevor man aufsteigen darf. Denn nur, weil ich die Grundschule geschafft habe, heißt das nicht, dass ich bereit für den Master bin. Keiner kann als Kind direkt nach der Erklärung zum Fahrradfahren losradeln und fährt nicht in den nächsten Baum oder kippt um.
	Üben beideutet die Anwendung des Skills in Ausreichender Menge oder Zeit für das Erreichen der kommenden Stufe. Das muss im Allgemeinen während des Spiels erfolgen. Allerdings können da die Lehrmeister - eingeschränkt - helfen. So könnte einem der Schlosser ein paar alte oder falsche oder einfach Übungsschlösser zur Verfügung stellen, damit man an ihnen üben kann. Das kostet natürlich. Macht man dieses Üben unter Aufsicht eines Meisters, dann kommt man schneller voran (man muss zB nicht 20 sondern nur 10 Schlösser knacken). ABER die Menge der Übung, die man bei einem Meister absolvieren kann, variiert je nach aktueller Stufe. So kann man im niedrigsten Bereich alle Übungseinheiten beim Meister absolvieren, doch will man Großmeister werden, dann vllt nur die ersten 20 Prozent...
	\item andere Skills wiederum gibt es nur in einer Steigerungsform: Man kann sie oder auch nicht und vielleicht kann man besser werden, aber nur in einem marginalen Rahmen und dann nur durch Anwendung (zB tanzen)
\end{itemize}

\section{Skills}
\begin{itemize}
	\item Idee für Erweiterung des Soldaten-Regenerationsskills: ein Punkt kann sein, dass er gelernt hat, sein Essen bedachter zu verspeisen und besser zu verwerten. Das führt zu einer Erhöhung der Heilung pro Sekunde
\end{itemize}

\section{Quest-Belohnungen}
\begin{itemize}
	\item viele der optionalen Quests sollen abhängig von ihrer Beschaffenheit dem Spieler wirklich auch eine tolle Belohnung geben: mehr Leben und "mehr" Mana und Resistenzen. Da wir keine Gegenstände haben, in die man solches mittels Verzauberungen legen könnte, ist das eine schöne Möglichkeit, sie dem Spieler trotzdem zur Verfügung zu stellen und ihn damit stärker werden zu lassen
	\item  blöd gesagt: wenn ich eine stinknormale Banditenbande ausnehme, habe ich ja Wunden erfahren, also weiß ich wieder besser mit meinen körpereigenen Regenerationskräften umzugehen und habe ab dann eine leicht erhöhte Regeneration (anstatt 2 Prozent pro h während Rasten oder Schlafen halt 2,5 Prozent oder so). Oder wenn ich in der Quest einem Alchemisten der Uni bei der Mischung und oder dem Ausprobieren eines Trankes helfen soll, steigt meine Resistenz ggü. Giften oder so was. Oder weil ich bei der Banditenbande oben ja gegen die gekämpft habe, habe ich nun mehr Erfahrung im kleinen Ausweichen, weshalb ich es schaffe, häufig weniger Schaden zu nehmen, aka mein Leben erhöht sich etc
	\item  die Rewards normaler Nebenquests sind in der Regel festgelegt von ihrem Wert und ihrer Stufe (bei Waffen und Rüstung). Denn der Bauer wird dir egal wie mächtig du bist, immer nur das gleiche bieten können als Bezahlung. Die Hauptquest, die der Spieler ja auch aktiv erstmal warten lassen soll, um die Nebenquests zu erkunden und zu machen,  wird mitleveln, damit es nicht wie in vielen Spielen ist, dass man sich erstmal an Nebenquests hocharbeitet und die HQ danach nur noch durchrennen und Oneshotten ist und man die Belohnung "das mächtige Schwert des Weisen" direkt zerstört, weil man keinen Platz im Inventar hat, es aber mittlerweile 20 Stufen später weniger wert ist, als der Rest im Rucksack...
\end{itemize}