\chapter{Karte \& Ökologie}
\href{http://www.weltenbau-wissen.de/sci-fi-erschaffen-fantasy-welt-erstellen-einstieg/}{Welt erstellen} \\
\href{http://www.weltenbau-wissen.de/2015/10/weltenbau-fragenkatalog-oekologie-biologie/}{Fragenkatalog} \\
\href{http://www.weltenbau-wissen.de/2015/01/weltenbau-mit-weltkarte-karte-zeichnen-tutorial/}{Weltkarte zeichnen} \\
\href{https://inkarnate.com/}{Weltenbautool}

\section{Sonnensystem}
\subsection{Welt + Zwilling}
\begin{itemize}
	\item unsere Erde ist ein zwei-Planeten System - ähnlich wie wir es haben mit Erde und Mond. Entstehung auch ähnlich, allerdings ist unser "Mond" deutlich größer: beide Planeten haben fast die gleiche Masse und natürlich praktisch die gleichen Elemente
	\item die zwei Planeten sind kleiner als die Erde. Bzw. müssen sie halt mindestens so groß sein, dass sie eine Atmosphäre halten können: \\
	Es ist ein Problem, wenn unsere Planeten zu klein sind... Man geht davon aus, dass sich das Leben auf Planeten mit unterschiedlicher Größe auch deutlich anders entwickelt hätten (Gravitation ist hier ausschlaggebend). Tatsächlich bewegen sich alle als erd-ähnlich und potenziell bewohnbar eingestuften Planeten, die wir bisher entdeckt haben zischen 0,6 und 2,5 Erdmassen und einem Radius von 0,9 bis 1,4-fachen des Erdradius. Merkur hat 0,05 Erdmassen und nur einen Drittel des Erdradius, ist also viel viel zu klein um jemals Leben wie auf der Erde entwickeln zu können, selbst wenn er in der habitablen Zone läge. Der Mars hat übrigens 0,1 Erdmassen und den 0,5 fachen Durchmesser. Einzig die Venus läge mit 0,8 Erdmassen und etwa gleichen Radius in der vielleicht gerade noch so okayen Range. Wir werden also wohl oder übel mit fast-erdgleichen Planeten arbeiten müssen, zumindest wenn wir es korrekt halten wollen.
	\item dadurch, dass sie sich so dicht beieinander befinden, erfolgte nach der Entstehung des Lebens auf einem der Planeten die Übertragung auf den anderen durch Meteroiten-Einschläge, die ihrerseits Brocken in das All hinausschießen ließen, von denen halt welche mit Leben drauf auf den anderen Planeten auftrafen (da sie so dicht beieinander sind)
	\item im Laufe der Zeit ist das Gleiche erfolgt wie bei unserem Mond: die Drehung um die eigene Achse ist so, dass sich die Planeten immer mit der gleichen Seite angucken. Und die Drehung insgesamt im Doppel-Planeten-System ist jeweils auch so, dass es einen "normalen" Tag-Nacht-Rhythmus gibt
	\item "normal" heißt hier zB gleiche Stunden am Tag wie bei uns oder etwas mehr -- aaaber die Entfernung von der Sonne und die Bahn ist so, dass das Jahr deutlich länger ist. Wir haben also deutlich längere Jahreszeiten - was uns zugute kommt, wenn man bedenkt, dass wir keinen Winter und Herbst noch einfügen wollen :smile:
\end{itemize}

\section{Welt}
\section{Kontinent}
\begin{itemize}
	\item Ein Grenzreich ist ein kleiner bis mittlerer Staat, dem es wie Kambodscha unter den Roten iwas geht -> gerade mit sich selbst beschäftigt
\end{itemize}

\subsection{Gebirgszug}
\subsection{Gigantus Wald} \label{formation:gigantus}
\begin{itemize}
	\item wie bekannt, können auch Pflanzen die ihnen innewohnende Magie nutzen. Eine Art (!!) von Bäumen haben folgendes entwickelt: sie ermöglichen mittels der Magie die Wasserversorgung der oberen Baumabschnitte gegen die Gravitation. Das Größeneinschränkende Element normaler Bäume ist damit weg. Daher konnten die Bäume extrem groß werden, bevor weitere Prozesse ihr Wachstum stoppten und hatten damit den Platz an der Sonne sicher
	\item  diese Garganten (Working title) erreichen Höhen von 500m und ähnlich wie aus dem Dschungel bekannt, bilden sich somit verschiedene vertikale Lebensbereiche
	\item der Boden ist bedeckt mit allerlei Gehölz und Sträuchern, die wenig Sonnenlicht brauchen, und ganz vorne voran riesigen Pilzen, die in Symbiose mit den riesigen Bäumen ebenfalls ein gigantisches unterirdisches Netzwerk bildeten und nun gigantische Fruchtkörper (:mushroom:  das da) ausbilden können
	\item die ersten kräftigen Äste der Garganten haben stark grüne, mit viel Chlorophyll gefüllte Blätter, die jegliches Licht, das durch die oberen Schichten kommt, aufnehmen kann. Für unsere Verhältnisse stehen die Bäume zwar ewig weit auseinander, doch auf ein normales Größenverhältnis reduziert nicht. Ihre Äste können sich an den Spitzen erreichen und bilden so mittels der Äste und den riesigen Laubblättern eine Art zweiten Boden, beginnend in 200m Höhe
	\item durch die Ausmaße der Äste sammelt sich Erde hier und dort und es wachsen normale Bäume auf dieser Ebene, allerdings nicht zu weit vom Stamm der Garganten entfernt. Ebenso finden sich hier normale Büsche etc. Schlingpflanzen ziehen sich um die dicken Äste und es bildete sich so im Laufe der Zeit in dicker "Boden" - mit unerwartete Löchern hier und da
	\item Da wir uns hier schon in der Krone der Bäume befinden, ist ab dieser zweiten Schicht ein langsamer Verlauf in den folgenden 300m: Die Baumkrone begünstigt in ihren Ausmaßen das Wachsen anderer Bäume, Sträucher, Kräuter etc, insbesondere parasitischer Lebensformen wie Schlingpflanzen und Misteln. Auch einige "normale Bäume" wachsen nicht nur auf den Erdansammlungen, sondern auch in das Holz der Äste hinein. Insgesamt bildet sich so wie eine Art Gerüst, welches aus festen "Platformen", aus festen und wackligen Wegen horizontal, vertikal und schräg, aus Bereichen lockerer Vegetation (wie Lianen) und aus "Lichtungen" (3D, nicht 2D) besteht.
	\item nach oben hin werden die Blätter der Garganten immer lichtdurchlässiger. Die untersten Blätter sind mehrere Meter dick und dunkelgrün vor Chlorophyll. Die obersten sind nur ein paar Zentimeter dick, kleiner, leicht hellgrün gefärbt und ansonsten Lichtdurchlässig. Das sorgt dafür, dass das Licht bis unten hin durchkommt und effizient genutzt werden kann, um den riesigen Baum (und seine Parasiten) zu versorgen
	\item ebenso wie die Fauna, hat sich hier auch viel Getier angesammelt und angepasst in den verschiedenen Stufen. So nisten bestimmte Vögel nur in den oberen 50m der Baumkronen. Auch ein paar Menschen fanden sich hier vor langer Zeit ein und haben sich an das Leben in den Baumkronen (von 200-350m etwa) angepasst. Es ist die Heimat der \npref{rasse:sylvan}.
	\item allerdings ist dann doch der einschränkende faktor zum einen die stabilität von holz, bevor der baum unter der eigenlast zusammenbricht, und die biegsamkeit von holz, bevor höhenwinde den baum zerbrechen. gegen letzteres könnte der baum natürlich einfach sehr dick sein. mit werten, die ich jetzt auf die schnelle für holz gefunden hab, sollte das ganze bei wenigen hundert m höhe schon instabil werden. man könnte vielleicht argumentieren, dass der baum nicht nur wasser besser transportieren kann, sondern auch mineralstoffe, um seine stabilität zu erhöhen. dann hätte man gleich ein inhärent sehr hartes holz in der welt. bei dieser höhe ist die breite auch fast egal (solange sie wenige meter überschreitet)
\end{itemize}

\section{Land}
\section{Tal}
\subsection{Teich 1}
\subsection{Teich 2}
\subsection{Großer Wald}
\subsection{Klippe am Eingang}

