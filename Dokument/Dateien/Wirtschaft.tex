\chapter{Wirtschaft}
Relevante Links diesbezüglich: \\
\href{https://www.leben-im-mittelalter.net/kultur-im-mittelalter/wirtschaft/handwerk.html}{Handwerk} \\
\href{https://www.leben-im-mittelalter.net/alltag-im-mittelalter/arbeit-und-berufe/handwerker.html}{Handwerker} \\
\href{https://www.leben-im-mittelalter.net/alltag-im-mittelalter/arbeit-und-berufe/handwerker/handwerksberufe.html}{Handwerksberufe} \\
\href{https://www.leben-im-mittelalter.net/kultur-im-mittelalter/wirtschaft/handel.html}{Handel}\\
\href{https://www.leben-im-mittelalter.net/alltag-im-mittelalter/arbeit-und-berufe/bauern.html}{Bauern}\\


\begin{outline}
	\1 Rüstungsschmied
	\1 Waffenschmied
	\1 Allgemeinschmied
	\1 Goldschmied
	\1 Weber
	\1 Winzer
	\1 Bierbrauer
	\1 Jäger
	\1 Abdecker
\end{outline}

\section{Landwirtschaft}


\section{Handwerk}

\section{Handel}

\section{Tal}
\subsection{Berufe}
Eine Zeile heißt ein Beruf, der das alles verbindet. \\
Normale Bevölkerung:
\begin{outline}
	\1 Bauern \& Hirten \& Imker ("`Zeidler"')
	\1 Tavernenwirt \& Brauerei \& Bader \& Müller \& Bäcker
	\1 Schmied \& Wappner
	\1 Jäger
	\1 Köhler
	\1 Holzfäller
	\1 Sammler: Pilze, Kräuter, Beeren
	\1 Fischer \& Netzflicker \& Besenbinder u.ä. (bei schlechtem Wetter)
	\1 Knochenhauer (=Metzger)
	\1 Lederer \& Schuster \& Sattler \& Seiler
	\1 Gerber \& Seifensieder \& Abdecker/Vasner
	\1 Totengräber
	\1 -> die Frauen von Männern mit bestimmten Berufen wie zB Jäger übernehmen die Textilarbeit für das restliche Dorf, Hebamme, ...
\end{outline}

Vom Orden:
\begin{outline}
	\1 Geistlicher mit Familie, dient auch als Schulze/Verweser
	\1 evtl. 2 weitere niedere Geistliche, zB für Grundbildung, etwas Magie?
	\1 eine Einheit Soldaten: Schutz vor Wölfen, Schutz des Gesetzes
\end{outline}
