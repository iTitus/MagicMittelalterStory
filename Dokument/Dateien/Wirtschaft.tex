\chapter{Wirtschaft}
Relevante Links diesbezüglich: \\
\href{https://www.leben-im-mittelalter.net/kultur-im-mittelalter/wirtschaft/handwerk.html}{Handwerk} \\
\href{https://www.leben-im-mittelalter.net/alltag-im-mittelalter/arbeit-und-berufe/handwerker.html}{Handwerker} \\
\href{https://www.leben-im-mittelalter.net/alltag-im-mittelalter/arbeit-und-berufe/handwerker/handwerksberufe.html}{Handwerksberufe} \\
\href{https://www.leben-im-mittelalter.net/kultur-im-mittelalter/wirtschaft/handel.html}{Handel}\\
\href{https://www.leben-im-mittelalter.net/alltag-im-mittelalter/arbeit-und-berufe/bauern.html}{Bauern}\\


\begin{itemize}
	\item Rüstungsschmied
	\item Waffenschmied
	\item Allgemeinschmied
	\item Goldschmied
	\item Weber
	\item Winzer
	\item Bierbrauer
	\item Jäger
	\item Abdecker
\end{itemize}

\section{Landwirtschaft}


\section{Handwerk}

\section{Handel}

\section{Tal}
\subsection{Berufe}
Eine Zeile heißt ein Beruf, der das alles verbindet. \\
Normale Bevölkerung:
\begin{itemize}
	\item Bauern \& Hirten \& Imker ("`Zeidler"')
	\item Tavernenwirt \& Brauerei \& Bader \& Müller \& Bäcker
	\item Schmied \& Wappner
	\item Jäger
	\item Köhler
	\item Holzfäller
	\item Sammler: Pilze, Kräuter, Beeren
	\item Fischer \& Netzflicker \& Besenbinder u.ä. (bei schlechtem Wetter)
	\item Knochenhauer (=Metzger)
	\item Lederer \& Schuster \& Sattler \& Seiler
	\item Gerber \& Seifensieder \& Abdecker/Vasner
	\item Totengräber
	\item -> die Frauen von Männern mit bestimmten Berufen wie zB Jäger übernehmen die Textilarbeit für das restliche Dorf, Hebamme, ...
\end{itemize}

Vom Orden:
\begin{itemize}
	\item Geistlicher mit Familie, dient auch als Schulze/Verweser
	\item evtl. 2 weitere niedere Geistliche, zB für Grundbildung, etwas Magie?
	\item eine Einheit Soldaten: Schutz vor Wölfen, Schutz des Gesetzes
\end{itemize}
