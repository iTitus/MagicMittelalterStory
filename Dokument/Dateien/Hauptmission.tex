\chapter{Hauptmission}
\section{Prolog}
Teil der Demo
\begin{itemize}
	\item Quests zur Festlegung der Magieausprägungen des Spielers
	\item Quests zur Erlernung von Fähigkeiten
	\item Lernen der Questarten
\end{itemize}
\section{Kapitel 1: Jugend}
Teil der Demo \\
Die drei Charas lernen sich kennen und werden Freunde.

\subsection{Schlüsselmoment}
Alle drei Hauptcharaktere haben einen speziellen Schlüsselmoment.

\subsubsection{Spionin}
Erwischt beim Klauen -> Totschlag -> wird gehängt -> überlebt und wird in den Geheimdienst aufgenommen

\subsubsection{Diplomatin}
Vom Blitz getroffen? -> mehrere Tage/Wochen im Koma -> kann beide vererbte Magie-Kontroll-Arten nutzen, also auch Elektrizität \\
Freunde dann bereits "`tot"' bzw. aus dem Tal raus zur Armee


\section{Kapitel 2: Ausbildung}
Tutorials zu den Fertigkeiten
\section{Kapitel 3: Kampf für den Orden}
erste Quests
\section{Kapitel 4: Forschung zur Magie in Gegenständen}
\begin{itemize}
	\item Spieler stößt auf Ritual/Erz wodurch man Gegenstände mit Magie verbinden kann
	\item Zuspitzung der Situation mit den Rebellen
\end{itemize}
\section{Kapitel 5: Erschaffung eines Gottes}
\begin{itemize}
	\item Quests gegen die Rebellen: Entscheidung für Erfüllung oder Missachtung der Aufträge bleibt offen
	\item Erschaffung des Gottes, was das Leben aller in einem näheren Umfeld aussaugt (schnelle Alterung), alle anderen sind ihrer Magie beraubt allerdings wirkt auch keine Magie mehr auf sie
	\item Spieler wird zum Propheten des Gottes
\end{itemize}

\section{Easter Eggs \& schlechte Witze}
\begin{itemize}
	\item PC wieder zuhause: "Guten Tag, Vater. Ich bin hungrig. Was gibt's?" \\ Vater: "Hallo hungrig, ich bin dein Vater. Nicht so viel los hier..." \\ Erst im eigentlichen Spiel bei Heimatbesuch.
	\item  PC: "hallo, dürfte ich bitte mal eine Frage stellen?" \\ NSC: "hast du damit gerade schon" \\ PC: "Okay, noch eine?" \\ NSC: "hast  du wieder gerade" \\ PC: "okay, noch zwei?" \\ NSC: "hast du auch schon" \\ PC: "hä, wann?" \\ NSC: "jetzt"
\end{itemize}