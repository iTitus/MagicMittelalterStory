\chapter{Quests}

\section{Arten}

\section{Belohnungen}
\begin{itemize}
	\item viele der optionalen Quests sollen abhängig von ihrer Beschaffenheit dem Spieler wirklich auch eine tolle Belohnung geben: mehr Leben und "mehr" Mana und Resistenzen. Da wir keine Gegenstände haben, in die man solches mittels Verzauberungen legen könnte, ist das eine schöne Möglichkeit, sie dem Spieler trotzdem zur Verfügung zu stellen und ihn damit stärker werden zu lassen
	\item  blöd gesagt: wenn ich eine stinknormale Banditenbande ausnehme, habe ich ja Wunden erfahren, also weiß ich wieder besser mit meinen körpereigenen Regenerationskräften umzugehen und habe ab dann eine leicht erhöhte Regeneration (anstatt 2 Prozent pro h während Rasten oder Schlafen halt 2,5 Prozent oder so). Oder wenn ich in der Quest einem Alchemisten der Uni bei der Mischung und oder dem Ausprobieren eines Trankes helfen soll, steigt meine Resistenz ggü. Giften oder so was. Oder weil ich bei der Banditenbande oben ja gegen die gekämpft habe, habe ich nun mehr Erfahrung im kleinen Ausweichen, weshalb ich es schaffe, häufig weniger Schaden zu nehmen, aka mein Leben erhöht sich etc
	\item  die Rewards normaler Nebenquests sind in der Regel festgelegt von ihrem Wert und ihrer Stufe (bei Waffen und Rüstung). Denn der Bauer wird dir egal wie mächtig du bist, immer nur das gleiche bieten können als Bezahlung. Die Hauptquest, die der Spieler ja auch aktiv erstmal warten lassen soll, um die Nebenquests zu erkunden und zu machen,  wird mitleveln, damit es nicht wie in vielen Spielen ist, dass man sich erstmal an Nebenquests hocharbeitet und die HQ danach nur noch durchrennen und Oneshotten ist und man die Belohnung "das mächtige Schwert des Weisen" direkt zerstört, weil man keinen Platz im Inventar hat, es aber mittlerweile 20 Stufen später weniger wert ist, als der Rest im Rucksack...
\end{itemize}